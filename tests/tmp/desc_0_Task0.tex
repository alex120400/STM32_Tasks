\documentclass[a4paper,12pt]{article}
\usepackage{a4wide}
\usepackage{tikz}
\usetikzlibrary{calc}

\usepackage[ngerman]{babel}
\usepackage{csquotes}

\usepackage{hyperref}
\usepackage[backend=biber, style=ieee, citestyle=numeric-comp, url=false, doi=false, isbn=false]{biblatex}
\usepackage{xurl}
\addbibresource{bib.bib}
\AtEveryBibitem{\clearfield{month}}
\AtEveryBibitem{\clearfield{day}}

\begin{document}
\pagestyle{empty}
\setlength{\parindent}{0em}
\section*{Task 1: LED Ansteuerung mittels PWM (Pulsweitenmodulation)}

Ihre Aufgabe ist es, eine der verfügbaren LEDs auf dem Adapter Board für das Nucleo Board über die Capture/Compare Module eines Timers zu steuern. Nutzen Sie hierzu die Informationen im Datenblatt \cite{data_sheet} und Reference Manual \cite{ref_manual} des STM32F334 sowie das User Manual für das Nucleo-64 Board \cite{nucleo_manual} und die Beschreibung der LL Driver \cite{driver_manual}. Verwenden Sie außerdem das bereits vorbereitete file \enquote{pwm.c} und beachten sie die folgenden Vorgaben für Ihre Lösung:

\begin{itemize}
\item Frequenz mit der die LED blinken soll: $f_{PWM}$= 0.5\,Hz ($=\frac{1}{T_{PWM}}$)
\item Tastverh\"altnis (duty cycle):  $\frac{T_{HIGH}}{T_{PWM}}$ = 46\,\%
\item Anzusteuernder Pin (und damit festgelegte LED): PA6
\item Zu nutzender Timer/Channel: TIM16_CH1
\item Die Anfangsflanke (wechsel zu High) des Signals soll konstant gehalten und die Endflanke (wechsel zu Low) entsprechend moduliert werden.
\end{itemize}
\vspace{0.3cm}

Es sind die Capture/Compare Kanäle des spezifischen Timers zu nutzen. Lösungen die etwaige Delays oder Interrupt Routinen nutzen werden nicht als Lösung akzeptiert.
\\



Programmieren Sie das obig definierte Verhalten in der angeh\"angten Datei \enquote{pwm.c}.
\\

Um Ihre L\"osung abzugeben, senden Sie ein E-Mail mit dem Betreff \enquote{Result Task 0} und Ihrer Datei \enquote{pwm.c}  an test@test.at.

\vspace{0.7cm}

Viel Erfolg und m\"oge die Macht mit Ihnen sein.

\printbibliography[heading=bibintoc]

\end{document}