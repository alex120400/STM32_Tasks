\documentclass[a4paper,12pt]{article}
\usepackage{a4wide}
\usepackage{tikz}
\usetikzlibrary{calc}

\usepackage[english]{babel}
\usepackage{csquotes}

\usepackage{hyperref}
\usepackage[backend=biber, style=ieee, citestyle=numeric-comp, url=false, doi=false, isbn=false]{biblatex}
\usepackage{xurl}
\addbibresource{bib.bib}
\AtEveryBibitem{\clearfield{month}}
\AtEveryBibitem{\clearfield{day}}

\begin{document}
\pagestyle{empty}
\setlength{\parindent}{0em}
\section*{Task {{TASKNR}}: Controlling the RGB LED using PWM (Pulse Width Modulation)}

Your task is to control the red and green channels of the RGB-LED on the adapter board for the Nucleo board using the capture/compare modules of two timers. Use the information in the datasheet \cite{data_sheet} and reference manual \cite{ref_manual} of the STM32F334 as well as the user manual for the Nucleo-64 board \cite{nucleo_manual} and the description of the LL driver \cite{driver_manual}. Also use the already prepared file \enquote{pwm\_rgb.c} and note the following specifications for your solution:

\begin{itemize}
\item Specifications for the red LED:
\begin{itemize}
\item Frequency: $f_{PWM, red}$= {{FRQred}}\,Hz ($=\frac{1}{T_{PWM_red}}$)
\item Duty cycle: $\frac{T_{HIGH,red}}{T_{PWM,red}}$ = {{DUTYred}}\,\%
\item Pin: PC7
\item Timer/Channel: Timer 3, Channel 2
\end{itemize}
\item Specifications for the green LED:
\begin{itemize}
\item Frequency: $f_{PWM, green}$= {{FRQgreen}}\,Hz ($=\frac{1}{T_{PWM_green}}$)
\item Duty cycle: $\frac{T_{HIGH,green}}{T_{PWM,green}}$ = {{DUTYgreen}}\,\%
\item Pin: PA7
\item Timer/Channel: Timer 17, Channel 1
\end{itemize}
\item The rising edge (change to High) of the signals should be kept constant and the falling edge (change to Low) should be modulated accordingly.
\end{itemize}
\vspace{0.3cm}

The capture/compare channels of the specific timers must be used. Solutions that use any delays or interrupt routines will not be accepted as a solution.
\\

\fbox{\parbox{\linewidth}{Attention: Timers 16 and 17 have complementary (N) channels, which together with the regular channels and the introduction of a dead-time, ensure a defined sequence of e.g. switching signals. For this reason, with these timers, another bit (the MOE-bit) in the TIMx\_BDTR register must be set to generate an output. Use the command \enquote{LL\_TIM\_EnableAllOutputs(TIMx)} for this purpose if you are to use one of the two timers and after you have configured the timer.}}
\\

Program the above-defined behavior in the attached file \enquote{pwm\_rgb.c}.
\\

To submit your solution, send an email with the subject \enquote{Result Task {{ TASKNR }}} and your file \enquote{pwm\_rgb.c} to {{ SUBMISSIONEMAIL }}.

\vspace{0.7cm}

Good luck and may the force be with you.

\newpage
\printbibliography[heading=bibintoc]

\end{document}