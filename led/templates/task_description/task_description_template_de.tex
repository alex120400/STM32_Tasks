\documentclass[a4paper,12pt]{article}
\usepackage{a4wide}
\usepackage{tikz}
\usetikzlibrary{calc}

\usepackage[ngerman]{babel}

\begin{document}
\pagestyle{empty}
\setlength{\parindent}{0em}
\section*{PWM (Pulsweitenmodulation)}

Ihre Aufgabe ist es, das Verhalten einer Entity  namens "`pwm"' zu programmieren. Die Entity ist in der angeh\"angten Datei "`pwm.vhdl"' deklariert und hat folgende Eigenschaften:

\begin{itemize}
\item Eingang:  CLK vom Typ std\_logic; dies ist ein rechteckiges Taktsignal mit 50 MHz (Periode $T_{CLK}$= 20 ns)
\item Ausgang: O vom Typ std\_logic; dies ist der Port des Ausgangssignals
\end{itemize}
\begin{center}
\begin{tikzpicture}
\draw node [draw,rectangle, minimum height=15mm, minimum width=35mm,rounded corners=2mm,thick](entity){};
\draw[->,thick] ($ (entity.west)-(10mm,0mm)$) -- ($ (entity.west) - (0mm,0mm)$);
\draw node at ($ (entity.west)-(15mm,0mm)$){CLK};

\draw[->,thick] (entity.east) -- ($ (entity.east) + (10mm,0)$);;
\draw node at ($ (entity.east) + (12mm,0)$){O};

\draw node at ($ (entity) - (0,12mm)$){pwm};

\end{tikzpicture}
\end{center}

Ver\"andern sie die Datei "`pwm.vhdl"' nicht!\\

Die Entity "`pwm"' soll ein pulsweitenmoduliertes Signal (PWM) aus einem Taktsignal (Eingangsport CLK) erzeugen und dabei folgende Eigenschaften erf\"ullen:
\begin{itemize}
\item Periode $T_{PWM}$ = {{PERIOD}}, Frequenz $f_{PWM}$= {{FRQ}}
\item Tastverh\"altnis (duty cycle)  $\frac{T_{HIGH}}{T_{PWM}}$ = {{DUTY}}
\item Die Anfangsflanke des Signals soll konstant gehalten und die Endflanke entsprechend moduliert werden.
\end{itemize}
\vspace{0.3cm}

Programmieren Sie dieses Verhalten in der angeh\"angten Datei "`pwm\_beh.vhdl"'.
\\

Um auf einem FPGA korrekt synthetisiert werden zu k\"onnen, muss die Erzeugung des PWM Signals ohne die VHDL Constructs 'after' und 'wait' geschehen.
\\

Um Ihre L\"osung abzugeben, senden Sie ein E-Mail mit dem Betreff "`Result Task {{ TASKNR }}"' und Ihrer Datei "`pwm\_beh.vhdl"'  an {{ SUBMISSIONEMAIL }}.

\vspace{0.7cm}

Viel Erfolg und m\"oge die Macht mit Ihnen sein.

\end{document}
