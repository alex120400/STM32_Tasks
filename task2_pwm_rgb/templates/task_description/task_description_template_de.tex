\documentclass[a4paper,12pt]{article}
\usepackage{a4wide}
\usepackage{tikz}
\usetikzlibrary{calc}

\usepackage[ngerman]{babel}
\usepackage{csquotes}

\usepackage{hyperref}
\usepackage[backend=biber, style=ieee, citestyle=numeric-comp, url=false, doi=false, isbn=false]{biblatex}
\usepackage{xurl}
\addbibresource{bib.bib}
\AtEveryBibitem{\clearfield{month}}
\AtEveryBibitem{\clearfield{day}}

\begin{document}
\pagestyle{empty}
\setlength{\parindent}{0em}
\section*{Task {{TASKNR}}: Ansteuerung der RGB LED mittels PWM (Pulsweitenmodulation)}

Ihre Aufgabe ist es, den roten und grünen Kanal der RGB-LED auf dem Adapter Board für das Nucleo Board über die Capture/Compare Module zweier Timers zu steuern. Nutzen Sie hierzu die Informationen im Datenblatt \cite{data_sheet} und Reference Manual \cite{ref_manual} des STM32F334 sowie das User Manual für das Nucleo-64 Board \cite{nucleo_manual} und die Beschreibung der LL Driver \cite{driver_manual}. Verwenden Sie außerdem die bereits vorbereitete Datei \enquote{pwm\_rgb.c} und beachten sie die folgenden Vorgaben für Ihre Lösung:

\begin{itemize}
\item Vorgaben für die rote LED:
\begin{itemize}
    \item Frequenz: $f_{PWM, red}$= {{FRQred}}\,Hz ($=\frac{1}{T_{PWM_red}}$)
\item Tastverh\"altnis (duty cycle):  $\frac{T_{HIGH,red}}{T_{PWM,red}}$ = {{DUTYred}}\,\%
\item Pin: PC7
\item Timer/Channel: Timer 3, Channel 2
\end{itemize}
\item  Vorgaben für die grüne LED:
\begin{itemize}
    \item Frequenz: $f_{PWM, green}$= {{FRQgreen}}\,Hz ($=\frac{1}{T_{PWM_green}}$)
\item Tastverh\"altnis (duty cycle):  $\frac{T_{HIGH,green}}{T_{PWM,green}}$ = {{DUTYgreen}}\,\%
\item Pin: PA7
\item Timer/Channel: Timer 17, Channel 1
\end{itemize}
\item Die Anfangsflanke (wechsel zu High) der Signale soll konstant gehalten und die Endflanke (wechsel zu Low) entsprechend moduliert werden.
\end{itemize}
\vspace{0.3cm}

Es sind die Capture/Compare Kanäle der spezifischen Timer zu nutzen. Lösungen die etwaige Delays oder Interrupt Routinen nutzen werden nicht als Lösung akzeptiert.
\\

\fbox{\parbox{\linewidth}{Achtung: Die Timer 16 und 17 besitzen komplementäre (N) Kanäle, die zusammen mit den gewöhnlichen Kanälen und der Einführung einer Tod-Zeit für eine definierte Abfolge an z.B. Schaltsignalen sorgen. Deshalb muss bei diesen Timern ein weiteres bit (das MOE-bit) im TIMx\_BDTR Register gesetzt werden um einen Output zu generieren. Verwenden Sie hierzu den Befehl \enquote{LL\_TIM\_EnableAllOutputs(TIMx)} sofern Sie einen der beiden Timer benutzen sollen und nachdem Sie den Timer konfiguriert haben.}}
\\

Programmieren Sie das obig definierte Verhalten in der angeh\"angten Datei \enquote{pwm\_rgb.c}.
\\

Um Ihre L\"osung abzugeben, senden Sie ein E-Mail mit dem Betreff \enquote{Result Task {{ TASKNR }}} und Ihrer Datei \enquote{pwm\_rgb.c}  an {{ SUBMISSIONEMAIL }}.

\vspace{0.7cm}

Viel Erfolg und m\"oge die Macht mit Ihnen sein.

\newpage
\printbibliography[heading=bibintoc]

\end{document}
