\documentclass[a4paper,12pt]{article}
\usepackage{a4wide}
\usepackage{tikz}
\usetikzlibrary{calc}

\usepackage[ngerman]{babel}
\usepackage{csquotes}

\usepackage{hyperref}
\usepackage[backend=biber, style=ieee, citestyle=numeric-comp, url=false, doi=false, isbn=false]{biblatex}
\usepackage{xurl}
\addbibresource{bib.bib}
\AtEveryBibitem{\clearfield{month}}
\AtEveryBibitem{\clearfield{day}}

\begin{document}
\pagestyle{empty}
\setlength{\parindent}{0em}
\section*{Task {{TASKNR}}: Extendend Interrupt Controller (EXTI) und Steuerung einer LED}

Ihre Aufgabe ist es, einen der verfügbaren Taster auf dem Adapter Board für das Nucleo Board über den Extendend Interrupt Controller (EXTI) zu handhaben und bei Betätigung des Tasters einen Interrupt auszulösen. Da der GPIO an dem der Taster hängt, über einen externen Pull-up Widerstand auf High gehoben ist, und bei Betätigung des Tasters auf GND gezogen wird, soll der Interrupt auch bei der fallenden Flanke des Signals ausgelöst werden. Bei Betätigung des Tasters soll dann eine gegebene LED eingeschalten werden. Wird zusätzlich der weitere Taster zuvor gedrückt und dann gehalten während der erste Taster betätigt wird, soll die gleiche LED wiederum ausgeschalten werden. Der zweite Taster soll über die Input-Funktionalitäten der GPIOs ausgelesen werden und nicht mittels EXTI gehandhabt werden. Nutzen Sie hierzu die Informationen im Datenblatt \cite{data_sheet} und Reference Manual \cite{ref_manual} des STM32F334 sowie das User Manual für das Nucleo-64 Board \cite{nucleo_manual} und die Beschreibung der LL Driver \cite{driver_manual}. Verwenden Sie außerdem die bereits vorbereitete Datei \enquote{exti\_led.c} und beachten sie die folgenden Vorgaben für Ihre Lösung:

\begin{itemize}
\item Anzusteuernder Pin (und damit festgelegte LED): {{PIN}}
\item Mittels EXTI zu steuernder Schalter und damit festgelegter Pin: {{SW}}
\item Die Anfangsflanke (wechsel zu Low) des Signals soll den Interrupt auslösen.
\end{itemize}
\vspace{0.3cm}

Es ist ausdrücklich die Interrupt-Steuerung mittels EXTI zu verwenden für den spezifizierten Taster und die GPIO-Input Funktionalität für den weiteren Taster. Lösungen über Polling werden nicht akzeptiert.
\\

Programmieren Sie das obig definierte Verhalten in der angeh\"angten Datei \enquote{exti\_led.c}.
\\

Um Ihre L\"osung abzugeben, senden Sie ein E-Mail mit dem Betreff \enquote{Result Task {{ TASKNR }}} und Ihrer Datei \enquote{exti\_led.c}  an {{ SUBMISSIONEMAIL }}.

\vspace{0.7cm}

Viel Erfolg und m\"oge die Macht mit Ihnen sein.
\\

\fbox{\parbox{\linewidth}{KI Anmerkung: Diese Programmieraufgaben stellen keine komplexen Aufgabenstellungen dar und sind daher relativ leicht von gängigen KI-Tools l\"osenbar. Diese Aufgaben haben nicht das Ziel komplexe, \textit{real-world} Problemstellungen wiederzuspiegeln, sondern sollen ein generelles Grundverständnis darüber schaffen, wie Embedded Systems zu programmieren sind und welche Grundkonzepte hier zu beachten sind. Werden moderne KI-Tools ohne jegliche Reflexion zur L\"osungsfindung herangezogen, wird dieses Ziel verfehlt. Desweiteren sind bei den Abschlussprüfungen keine KI-Tools erlaubt, sondern vielmehr ihre Lernerkenntnisse gefragt um zu \"uberprüfen ob Sie sich das erwartete Grundverständnis erarbeitet haben.}}
\\

\printbibliography[heading=bibintoc]

\end{document}
