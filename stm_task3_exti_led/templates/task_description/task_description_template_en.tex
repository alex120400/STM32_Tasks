\documentclass[a4paper,12pt]{article}
\usepackage{a4wide}
\usepackage{tikz}
\usetikzlibrary{calc}

\usepackage[english]{babel}
\usepackage{csquotes}

\usepackage{hyperref}
\usepackage[backend=biber, style=ieee, citestyle=numeric-comp, url=false, doi=false, isbn=false]{biblatex}
\usepackage{xurl}
\addbibresource{bib.bib}
\AtEveryBibitem{\clearfield{month}}
\AtEveryBibitem{\clearfield{day}}

\begin{document}
\pagestyle{empty}
\setlength{\parindent}{0em}
\section*{Task {{TASKNR}}: Extended Interrupt Controller (EXTI) and Controlling an LED}

Your task is to handle one of the available buttons on the adapter board for the Nucleo board using the Extended Interrupt Controller (EXTI) and trigger an interrupt when the button is pressed. Since the GPIO the button is connected to is pulled high via an external pull-up resistor and is pulled to GND when the button is pressed, the interrupt should be triggered on the falling edge of the signal. When the button is pressed, a specified LED should be switched on. If the second button is pressed and held before the first button is pressed, the same LED should be switched off. The second button should be read using the Input functionalities of the GPIOs and not handled via EXTI. Use the information in the datasheet \cite{data_sheet} and reference manual \cite{ref_manual} of the STM32F334 as well as the user manual for the Nucleo-64 board \cite{nucleo_manual} and the description of the LL driver \cite{driver_manual}. Also use the already prepared file \enquote{exti\_led.c} and note the following specifications for your solution:

\begin{itemize}
\item Pin to be controlled (and thus the specified LED): {{PIN}}
\item Switch to be controlled via EXTI and thus the specified Pin: {{SW}}
\item The rising edge (change to Low) of the signal should trigger the interrupt.
\end{itemize}
\vspace{0.3cm}

The interrupt control via EXTI must explicitly be used for the specified button, and the GPIO input functionality for the other button. Solutions using polling will not be accepted.
\\

Program the above-defined behavior in the attached file \enquote{exti\_led.c}.
\\

To submit your solution, send an email with the subject \enquote{Result Task {{ TASKNR }}} and your file \enquote{exti\_led.c} to {{ SUBMISSIONEMAIL }}.

\vspace{0.7cm}

Good luck and may the force be with you.
\\

\fbox{\parbox{\linewidth}{AI Note: These programming tasks do not represent complex problem statements and can therefore be solved relatively easily by common AI tools. The purpose of these tasks is not to reflect complex real-world scenarios, but rather to create a general foundational understanding of how embedded systems are programmed and which basic concepts must be considered. If modern AI tools are used for solving these tasks without any form of reflection, this objective is not achieved. Furthermore, AI tools are not permitted during the final examinations — instead, your own learning progress is required in order to verify that you have developed the expected fundamental understanding.}}
\\

\printbibliography[heading=bibintoc]

\end{document}