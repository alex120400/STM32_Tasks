\documentclass[a4paper,12pt]{article}
\usepackage{a4wide}
\usepackage{tikz}
\usetikzlibrary{calc}

\usepackage[ngerman]{babel}
\usepackage{csquotes}

\usepackage{hyperref}
\usepackage[backend=biber, style=ieee, citestyle=numeric-comp, url=false, doi=false, isbn=false]{biblatex}
\usepackage{xurl}
\addbibresource{bib.bib}
\AtEveryBibitem{\clearfield{month}}
\AtEveryBibitem{\clearfield{day}}

\begin{document}
\pagestyle{empty}
\setlength{\parindent}{0em}
\section*{Task 1: LED Ansteuerung mittels PWM (Pulsweitenmodulation)}

Ihre Aufgabe ist es, die User LED auf dem Nucleo Board über die Capture/Compare Module eines Timers zu steuern. Nutzen Sie hierzu die Informationen im Datenblatt \cite{data_sheet} und Reference Manual \cite{ref_manual} des STM32F334 sowie das User Manual für das Nucleo-64 Board \cite{nucleo_manual} und die Beschreibung der LL Driver \cite{driver_manual}. Nutzen Sie hierzu das bereits vorbereitete file \enquote{pwm.c} und beachten sie außerdem die folgenden Vorgaben:

\begin{itemize}
\item Periode mit der die LED blinken soll $T_{PWM}$ = {{PERIOD}}, Frequenz $f_{PWM}$= {{FRQ}}
\item Tastverh\"altnis (duty cycle)  $\frac{T_{HIGH}}{T_{PWM}}$ = {{DUTY}}
\item Die Anfangsflanke (wechsel zu High) des Signals soll konstant gehalten und die Endflanke (wechsel zu Low) entsprechend moduliert werden.
\end{itemize}
\vspace{0.3cm}

Programmieren Sie dieses Verhalten in der angeh\"angten Datei \enquote{pwm.c}.
\\

Es sind die Capture/Compare Kanäle des spezifischen Timers der mit der User LED verbunden ist zu nutzen. Lösungen die etwaige delays oder Interrupt Routinen nutzen werden nicht als Lösung akzeptiert.
\\

Um Ihre L\"osung abzugeben, senden Sie ein E-Mail mit dem Betreff \enquote{Result Task {{ TASKNR }}} und Ihrer Datei \enquote{pwm.c}  an {{ SUBMISSIONEMAIL }}.

\vspace{0.7cm}

Viel Erfolg und m\"oge die Macht mit Ihnen sein.

\printbibliography[heading=bibintoc]

\end{document}
